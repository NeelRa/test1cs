\documentclass{article}

\usepackage{pgfplots}
\usepackage[margin=0.75in, paperwidth=8.5in, paperheight=11in]{geometry}
\usepackage{setspace}
\usepackage{fancyvrb} % extended verbatim environments
\usepackage{framed}%To get shade behind text

\definecolor{shadecolor}{rgb}{0.9,0.9,0.9}%setting shade color


\begin{document}
\pagenumbering{gobble}

\doublespacing
\textbf{IB Computer Science }                        %%%(class number and section) 
 \hfill                             %%%(date of test)
$ {\bf Name: } \underline {Neel Ramani}{\hspace{2.5in}}$(3 points)

\begin{centering}
\vspace{1cm}
\textbf{Exam 1}\\
\end{centering}
\vspace{1cm}
 
$\bf{1)}$ Determine what each of the following Python expressions will return.  In other words, if these expressions were entered into the Python terminal, what would they return?
(5 points each)

\vspace{1cm}
  
 a.  
 \begin{verbatim}
 		5/2
		
		>>> 2
 \end{verbatim}
 
 b.   \begin{verbatim}
 		5./2.
		
		>>> 2.5
 \end{verbatim}
  \vspace{1cm}
 
 c.  
  \begin{verbatim}
 		5%3
		
		>>> 2
 \end{verbatim}
 \vspace{1cm}
  
 d. 
  \begin{verbatim}
 		not(5>6) and (True or False) 
		
	 	 >>> True
 \end{verbatim}
 \vspace{1cm}
 
 e. 
  \begin{verbatim}
 		(5==4) or (not True) 
		
		>>> False
 \end{verbatim}
 \vspace{1cm}

  \newpage
  
 $\bf{2)}$ Write the output of the following programs. (8 points each)
 
 \vspace{1cm}

 a.   \begin{verbatim}
 for i in range(3):
       print i*i
 print "hi"
 
>>> 0
    1
    4
    hi
 \end{verbatim}
 \vspace{1cm}
 
 b.  \begin{verbatim}
 s=0
 for x in [5,3,1]:
       s=s+x
 print s
 
 >>> 9
 \end{verbatim}
 \vspace{1cm}
 
 c.  \begin{verbatim}
 x=16
 while x  > 5:
       x=x/2
 print x
 
 >>> 4
 \end{verbatim}
 \vspace{1cm}
 
 d.   \begin{verbatim}
 a=7
 if a%2==1:
       print "yoda"
 else:
       print "do yoga"
       
 >>>  yoda      
 \end{verbatim}
 \vspace{1cm}
  \newpage

  
  $\bf{3)}$  The following questions are about Git. (10 points each)
  \vspace{0.5cm}
    
  a.  Explain how to create a new git repository.  Include all terminal commands and things you must do on github.  Assume your github user name is "Charlie" and your project is in a folder named "Project" in your Documents folder.  Name the repository "ProjRepo".
   
   \begin{verbatim}
Things to do on GitHub:
    a) Go to https://github.com
    b)Login to your account with the account name Charlie  and the password
    c)Click on a green button with a sign "+" on it, which will create a new repository
    d)Name the repository "ProjRepo"
    
Make Sure you have GitHub installed on your respective operative system before moving to the further steps.

Go to Terminal or to command bash of the particular operating system and type in the following commands:

$cd Documents
$cd Project
$git init
$git add .
$git commit -m "comment"
$git remote add origin https://github.com/Charlie/ProjRepo.git
$git push -u master origin 
   \end{verbatim}
  
  b.  Explain how to clone a repository name "Awesome" from github user named "Barry22".  Clone the repository into your Documents folder.
  \begin{verbatim}

1) Open Terminal,
2) Type in the following commands:
    $cd Documents
    $git clone https://github.com/Barry22/Awesome.git
  
  \end{verbatim}   
  
  \newpage
  
  $\bf{4)}$ Write a program that constructs an array filled with all of the prime numbers between 2 and 100. (20 points)
  
  \begin{verbatim}
  
a=[2]
num=3

while len(a)<100:
    prime=True
    for x in range(len(a)):
        if num%a[x]==0:
            prime=False
    if prime:
        a=a+[num]

    num=num+1

print a
  
  \end{verbatim}
  
  

 
\end{document}